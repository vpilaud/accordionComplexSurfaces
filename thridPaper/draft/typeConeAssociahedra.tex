\documentclass{amsart}

\usepackage[T1]{fontenc}
\usepackage{enumerate, amsmath, amsfonts, amssymb, amsthm, mathrsfs, wasysym, graphics, graphicx, xcolor, url, hyperref, hypcap, xargs, multicol, pdflscape, multirow, hvfloat, array, ae, aecompl, pifont, mathtools, a4wide, float, blkarray}
\usepackage[noabbrev,capitalise]{cleveref}
\usepackage{marginnote}
\hypersetup{colorlinks=true, citecolor=darkblue, linkcolor=darkblue}
\usepackage[all]{xy}
\usepackage{tikz}
\usepackage{tkz-graph}
\usetikzlibrary{trees, decorations, decorations.markings, shapes, arrows, matrix, calc, fit, intersections, patterns, angles}
\graphicspath{{figures/}}
\makeatletter\def\input@path{{figures/}}\makeatother
\usepackage{caption}
\captionsetup{width=\textwidth}

%%%%%%%%%%%%%%%%%%%%%%%%%%%%%%%%%%%%%%

% theorems
\newtheorem{theorem}{Theorem}[section]
\newtheorem{corollary}[theorem]{Corollary}
\newtheorem{proposition}[theorem]{Proposition}
\newtheorem{lemma}[theorem]{Lemma}
\newtheorem{conjecture}[theorem]{Conjecture}
\newtheorem*{theorem*}{Theorem}%[section]

\theoremstyle{definition}
\newtheorem{definition}[theorem]{Definition}
\newtheorem{example}[theorem]{Example}
\newtheorem{remark}[theorem]{Remark}
\newtheorem{question}[theorem]{Question}
\newtheorem{notation}[theorem]{Notation}
\newtheorem{assumption}[theorem]{Assumption}
\newtheorem{convention}[theorem]{Convention}

\crefname{equation}{Equation}{Equations}

% math special letters
\newcommand{\R}{\mathbb{R}} % reals
\newcommand{\N}{\mathbb{N}} % naturals
\newcommand{\Z}{\mathbb{Z}} % integers
\newcommand{\C}{\mathbb{C}} % complex
\newcommand{\I}{\mathbb{I}} % set of integers
\newcommand{\HH}{\mathbb{H}} % hyperplane
\newcommand{\K}{k} % field
\newcommand{\fA}{\mathfrak{A}} % alternating group
\newcommand{\fS}{\mathfrak{S}} % symmetric group
\newcommand{\cA}{\mathcal{A}} % algebra
\newcommand{\cC}{\mathcal{C}} % collection
\newcommand{\cS}{\mathcal{S}} % ground set
\newcommand{\uR}{\underline{R}} % underline set
\newcommand{\uS}{\underline{S}} % underline set
\newcommand{\uT}{\underline{T}} % underline set
\newcommand{\oS}{\overline{S}} % overline set
\newcommand{\ucS}{\underline{\cS}} % underline ground set
\renewcommand{\b}[1]{{\boldsymbol{#1}}} % bold letters
\newcommand{\h}{\widehat} % hat letters

% math commands
\newcommand{\set}[2]{\left\{ #1 \;\middle|\; #2 \right\}} % set notation
\newcommand{\bigset}[2]{\big\{ #1 \;\big|\; #2 \big\}} % big set notation
\newcommand{\Bigset}[2]{\Big\{ #1 \;\Big|\; #2 \Big\}} % Big set notation
\newcommand{\setangle}[2]{\left\langle #1 \;\middle|\; #2 \right\rangle} % set notation
\newcommand{\ssm}{\smallsetminus} % small set minus
\newcommand{\dotprod}[2]{\left\langle \, #1 \; \middle| \; #2 \, \right\rangle} % dot product
\newcommand{\symdif}{\,\triangle\,} % symmetric difference
\newcommand{\one}{{1\!\!1}} % the all one vector
\newcommand{\eqdef}{\mbox{\,\raisebox{0.2ex}{\scriptsize\ensuremath{\mathrm:}}\ensuremath{=}\,}} % :=
\newcommand{\defeq}{\mbox{~\ensuremath{=}\raisebox{0.2ex}{\scriptsize\ensuremath{\mathrm:}} }} % =:
\newcommand{\simplex}{\triangle} % simplex
\renewcommand{\implies}{\Rightarrow} % imply sign
\newcommand{\transpose}[1]{{#1}^t} % transpose matrix

% operators
\DeclareMathOperator{\conv}{conv} % convex hull
\DeclareMathOperator{\vect}{vect} % linear span
\DeclareMathOperator{\cone}{cone} % cone hull
\DeclareMathOperator{\inv}{inv} % inversions
\DeclareMathOperator{\ascents}{asc} % ascents
\DeclareMathOperator{\descents}{des} % descents

% others
\newcommand{\ie}{\textit{i.e.}~} % id est
\newcommand{\eg}{\textit{e.g.}~} % exempli gratia
\newcommand{\Eg}{\textit{E.g.}~} % exempli gratia
\newcommand{\apriori}{\textit{a priori}} % a priori
\newcommand{\viceversa}{\textit{vice versa}} % vice versa
\newcommand{\versus}{\textit{vs.}~} % versus
\newcommand{\aka}{\textit{a.k.a.}~} % also known as
\newcommand{\perse}{\textit{per se}} % per se
\newcommand{\ordinal}{\textsuperscript{th}} % th for ordinals
\newcommand{\ordinalst}{\textsuperscript{st}} % st for ordinals
\definecolor{darkblue}{rgb}{0,0,0.7} % darkblue color
\definecolor{green}{RGB}{57,181,74} % green color
\definecolor{violet}{RGB}{147,39,143} % violet color
\newcommand{\red}{\color{red}} % red command
\newcommand{\green}{\color{green}} % green command
\newcommand{\darkblue}{\color{darkblue}} % darkblue command
\newcommand{\defn}[1]{\textsl{\darkblue #1}} % emphasis of a definition
\newcommand{\para}[1]{\medskip\noindent\textit{#1.}} % paragraph
\renewcommand{\topfraction}{1} % possibility to have one page of pictures
\renewcommand{\bottomfraction}{1} % possibility to have one page of pictures
\newcommand{\ex}{_{\textrm{exm}}} % examples
\newcommand{\pa}{_{\textrm{pa}}} % path
\newcommand*\circled[1]{\tikz[baseline=(char.base)]{\node[shape=circle,draw,inner sep=1pt, scale=.7] (char) {#1};}}

% marginal comments
\usepackage{todonotes}
\newcommand{\arnau}[1]{\todo[color=orange!30]{#1 --- A.}}
\newcommand{\yann}[1]{\todo[color=red!30]{#1 --- Y.}}
\newcommand{\vincent}[1]{\todo[color=blue!30]{#1 \\ \hfill --- V.}}
\newcommand{\pierreguy}[1]{\todo[color=green!30]{#1 \\ \hfill --- PG.}}

% polytopes
\newcommandx{\Asso}[2][1=\bar Q,2={}]{\mathsf{Asso}^{#2}(#1)} % associahedron
\newcommandx{\Zono}[2][1=\bar Q,2={}]{\mathsf{Zono}^{#2}(#1)} % zonotope
\newcommand{\Fan}{\mathcal{F}} % fan
\newcommand{\multiplicityVector}{\b{m}} % multiplicity vector

% Type cone
\newcommand{\typeCone}{\mathbb{TC}} % type cone
\newcommand{\compatibilityDegree}[2]{(#1\,\|\,#2)} % compatibility degree
\newcommandx{\coefficient}[3][1={\ray[s]}, 2=\ray, 3=\ray']{\alpha_{#2,#3}(#1)} % coefficient in linear dependence

% Graph associahedra
\newcommand{\ground}{\mathrm{V}} % ground set
\newcommandx{\graphG}[1][1=G]{\mathrm{#1}} % graph
\newcommandx{\tube}[1][1=t]{\mathsf{#1}} % tube
\newcommandx{\tubes}{\mathcal{T}} % all tubes
\newcommandx{\tubing}[1][1=T]{\mathsf{#1}} % tubing
\newcommand{\connectedComponents}{\kappa} % connected components
\newcommand{\nestedComplex}{\mathcal{N}} % nested complex
\newcommand{\nestedFan}{\mathcal{F}} % nested fan
\newcommand{\nonDisconnecting}{\mathrm{nd}} % non-disconnecting vertices

% SPECIFIC NON-KISSING COMPLEX

% COMBINATORICS
% quiver
\newcommand{\blossom}{^\text{\ding{96}}} % blossom

% walks
\newcommand{\peaks}[1]{\mathsf{peaks}(#1)} % peaks
\newcommand{\deeps}[1]{\mathsf{deeps}(#1)} % deeps
\newcommand{\distinguishedWalk}[2]{\mathsf{dw}(#1,#2)} % distinguished walk
\newcommand{\distinguishedArrows}[2]{\mathsf{da}(#1,#2)} % distinguished arrows
\newcommand{\distinguishedString}[2]{\mathsf{ds}(#1,#2)} % distinguished string
\newcommand{\distinguishedSign}[2]{\varepsilon(#1,#2)} % distinguished sign

% non-kissing complex, lattice, flip graph
\newcommand{\kn}{\kappa} % kissing number
\newcommand{\KN}{\textsc{kn}} % Kissing Number
\newcommandx{\NKC}[1][1=\bar Q]{\mathcal{K}_{\mathrm{nk}}(#1)} % non-kissing complex
\newcommandx{\RNKC}[1][1=\bar Q]{\mathcal{C}_{\mathrm{nk}}(#1)} % non-kissing complex
\newcommandx{\NKL}[1][1=\bar Q]{\mathcal{L}_{\mathrm{nk}}(#1)} % non-kissing lattice
\newcommandx{\NKG}[1][1=\bar Q]{\mathcal{G}_{\mathrm{nk}}(#1)} % non-kissing flip graph
\newcommandx{\NFC}[1][1=\bar Q]{\mathcal{C}_{\mathrm{nf}}(#1)} % non-friendly complex
\newcommand{\peak}{\mathrm{peak}} % bottom dissection
\newcommand{\deep}{\mathrm{deep}} % top dissection
\newcommand{\reversed}[1]{#1^{\mathrm{rev}}} % reverse all arrows
\renewcommand{\top}{\mathrm{top}} % top
\newcommand{\bottom}{\mathrm{bot}} % bottom
\newcommand{\walk}{\operatorname{\omega}} % walk (of a curve)

% accordion and slalom complexes
\newcommand{\surface}{\mathcal{S}} % an orientable surface
\newcommand{\dual}{^*} % duality
\newcommand{\dissection}{\mathrm{D}} % dissection of a marked surface
\newcommand{\vertices}{\mathcal{V}} % vertices of a dissection
\newcommand{\edges}{\mathcal{E}} % edges of a dissection
\newcommand{\faces}{\mathcal{F}} % faces of a dissection
\renewcommandx{\AC}[1][1=\dissection]{\mathcal{K}_{\mathrm{acc}}(#1)} % accordion complex
\newcommandx{\RAC}[1][1=\dissection]{\mathcal{C}_{\mathrm{acc}}(#1)} % reduced accordion complex
\newcommandx{\SC}[1][1=\dissection\dual]{\mathcal{K}_{\mathrm{sla}}(#1)} % slalom complex
\newcommandx{\RSC}[1][1=\dissection\dual]{\mathcal{C}_{\mathrm{sla}}(#1)} % reduced slalom complex
\newcommandx{\NCC}[1][1={\dissection, \dissection\dual}]{\mathcal{K}_{\mathrm{nc}}(#1)} % non-crossing complex
\newcommandx{\RNCC}[1][1={\dissection, \dissection\dual}]{\mathcal{C}_{\mathrm{nc}}(#1)} % non-crossing complex
\newcommand{\curveof}{\operatorname{\gamma}} % curve (of a walk)
\newcommand{\edgeof}{\operatorname{\varepsilon}} % arc of a vertex
\newcommand{\dualedgeof}{\operatorname{\varepsilon}\dual} % dual arc of a vertex
\DeclareRobustCommand{\SSS}{\reflectbox{$\mathsf{Z}$}} % S
\DeclareRobustCommand{\ZZZ}{\mathsf{Z}} % Z
\newcommand{\vnext}[1]{#1_{\operatorname{next}}} % next point on the boundary of the surface
\newcommand{\vprevious}[1]{#1_{\operatorname{prev}}} % previous point on the boundary of the surface

% lattices
\newcommand{\meet}{\wedge} % meet
\newcommand{\join}{\vee} % join
\newcommand{\bigMeet}{\bigwedge} % meet
\newcommand{\bigJoin}{\bigvee} % join
\newcommand{\closure}[1]{#1^{\mathrm{cl}}} % closure operator
\newcommand{\coclosure}[1]{#1^{\mathrm{cocl}}} % coclosure operator
\newcommand{\uclosure}[1]{#1^{\underline{\mathrm{cl}}}} % loop free closure operator
\newcommand{\Bicl}[1]{\mathsf{Bic}(#1)} % biclosed sets
\newcommand{\projDown}{\pi_\downarrow} % down projection map
\newcommand{\projUp}{\pi^\uparrow} % up projection map
\newcommand{\JI}{\mathsf{JI}} % join-irreducible
\newcommand{\MI}{\mathsf{MI}} % meet-irreducible
\newcommand{\Cong}{\mathsf{Cong}} % congruence lattice
\newcommand{\con}{\mathrm{con}} % congruence contracting a cover relation
\newcommand{\ji}{\mathsf{ji}} % join irreducible
\newcommand{\mi}{\mathsf{mi}} % meet irreducible

% GEOMETRY
% d-vectors
\newcommand{\dvector}[1]{\mathbf{d}(#1)} % d-vector of the path #1
\newcommand{\dvectors}[1]{\mathbf{d}(#1)} % d-vectors of the paths #1
\newcommandx{\dvectorFan}[1][1=\bar Q]{\mathcal{F}^\mathbf{d}(#1)} % d-vector fan
% g-vectors
\newcommand{\gvector}[1]{\mathbf{g}(#1)} % g-vector of the path #1
\newcommand{\gvectors}[1]{\mathbf{g}(#1)} % g-vectors of the paths #1
\newcommandx{\gvectorFan}[1][1=\bar Q]{\mathcal{F}^\mathbf{g}(#1)} % g-vector fan
% c-vectors
\newcommand{\cvector}[2]{\mathbf{c}(#1 \in #2)} % c-vector of the path #1 in the facet #2
\newcommand{\cvectors}[1]{\mathbf{c}(#1)} % c-vectors of the paths #1
\newcommandx{\allcvectors}[1][1=\bar Q]{\mathbf{C}(#1)} % all c-vectors with respect to the initial cluster #1
\newcommandx{\cvectorFan}[1][1=\bar Q]{\mathcal{F}^\mathbf{c}(#1)} % fan of hyperplanes orthogonal to all c-vectors
% points, hyperplanes, half-spaces
\newcommand{\point}[1]{\mathbf{p}(#1)} % vertex of the grid associahedron corresponding to the cluster #1
\newcommandx{\ray}[1][1=r]{\mathbf{#1}} % ray
\newcommandx{\rays}[1][1=R]{\mathbf{#1}} % rays
\newcommand{\hs}{\mathbf{H}^{\le}} % half space
\newcommand{\HS}[1]{\mathbf{H}^{\le}(#1)} % half space
\newcommand{\hyp}{\mathbf{H}^{=}} % hyperplane
\newcommand{\Hyp}[1]{\mathbf{H}^{=}(#1)} % hyperplane
\newcommand{\fix}[1]{\mathrm{Fix}(#1)} % fix space

% ALGEBRA
\newcommand{\Hom}[1]{\operatorname{{\rm Hom}}_{#1}}
\newcommand{\Ext}[1]{\operatorname{{\rm Ext}}_{#1}}
\newcommandx{\AR}[1][1=\bar Q]{\mathrm{AR}(#1)} % Auslander-Reiten quiver
\newcommandx{\tTC}[1][1=\bar Q]{\mathcal{K}^{\textrm{s$\tau$-tilt}}(#1)} % tau-tilting complex
\newcommand{\rep}{\operatorname{{\rm rep}}}
\newcommand{\proj}{\operatorname{{\rm proj}}}
\newcommand{\koszul}{^!} % duality

%% formating the part command
%\makeatletter
%\def\part{\@startsection{part}{1}%
%\z@{.7\linespacing\@plus\linespacing}{.8\linespacing}%
%{\LARGE\sffamily\centering}}
%\@addtoreset{section}{part}
%\makeatother
%\renewcommand{\thesection}{\arabic{part}.\arabic{section}}

% formating the table of contents
\makeatletter
\def\l@section{\@tocline{1}{2pt}{0pc}{}{}}
\makeatother
\let\oldtocpart=\tocpart
\renewcommand{\tocpart}[2]{\bf\large\oldtocpart{#1}{#2}}
\let\oldtocsection=\tocsection
\renewcommand{\tocsection}[2]{\bf\oldtocsection{#1}{#2}}

%%%%%%%%%%%%%%%%%%%%%%%%%%%%%%%%%%%%%%

\title{Type cones of $\b{g}$-vector fans}

\thanks{YP, VP and PGP were partially supported by the French ANR grant SC3A~(15\,CE40\,0004\,01). AP and VP were partially supported by the French ANR grant CAPPS~(17\,CE40\,0018).}

\author{Arnau Padrol}
\address[Arnau Padrol]{IMJ - Paris Rive Gauche, Sorbonne Universit\'e}
\email{arnau.padrol@imj-prg.fr.}
\urladdr{\url{https://webusers.imj-prg.fr/~arnau.padrol/}}

\author{Yann Palu}
\address[Yann Palu]{LAMFA, Universit\'e Picardie Jules Verne, Amiens}
\email{yann.palu@u-picardie.fr}
\urladdr{\url{http://www.lamfa.u-picardie.fr/palu/}}

\author{Vincent Pilaud}
\address[Vincent Pilaud]{CNRS \& LIX, \'Ecole Polytechnique, Palaiseau}
\email{vincent.pilaud@lix.polytechnique.fr}
\urladdr{\url{http://www.lix.polytechnique.fr/~pilaud/}}

\author{Pierre-Guy Plamondon}
\address[Pierre-Guy Plamondon]{Laboratoire de Math\'ematiques d'Orsay, Universit\'e Paris-Sud, CNRS, Universit\'e Paris-Saclay}
\email{pierre-guy.plamondon@math.u-psud.fr}
\urladdr{\url{https://www.math.u-psud.fr/~plamondon/}}

%%%%%%%%%%%%%%%%%%%%%%%%%%%%%%%%%%%%%%

\begin{document}

\begin{abstract}
We investigate the type cone of the $\b{g}$-vector fans for different generalizations of the associahedron (hopefully generalized associahedra, gentle associahedra, graph associahedra, and maybe quotientopes and brick polytopes).
The objective is to determine which are the exchangeable pairs of objects that describe the type cone.
When the type cone happens to be simplicial, we derive simple descriptions of all polytopal realizations of the $\b{g}$-vector fan.
\end{abstract}

\maketitle

We explain recent results of~\cite{BazierMatteDouvilleMousavandThomasYildirim}. Some advantages of our approach are:
\begin{itemize}
\item We manage to extend their results to other families of $\b{g}$-vector fans: first to all $\b{g}$-vector fans of finite type cluster algebras acyclic or not (whose polytopality was only proved recently in~\cite{HohlwegPilaudStella}), then to all $\b{g}$-vector fans of the $\tau$-tilting finite gentle algebras (whose polytopality was proved recently in~\cite{PaluPilaudPlamondon}), and finally to the nested fans of graph associahedra (whose polytopality was studied in~\cite{CarrDevadoss, Devadoss, Postnikov, FeichtnerSturmfels, Zelevinsky}).
\item We have a simpler proof, only based on the transformation between the classical descriptions of a polytope of the form~$\set{\b{x} \in \R^n}{\b{G}\b{x} \le \b{h}}$ and~$\set{\b{z} \in \R^N}{\b{K}\b{z} = \b{K}\b{h} \text{ and } \b{z} \ge 0}$ (this transformation is standard in optimization). In particular, we do not need to consider only rational polytopes and pass to the limit to deal with all real descriptions.
\item We observe that we obtain all realizations of the $\b{g}$-vector fans of the considered families.
\end{itemize}


\section{Polytopal realizations and type cone of a simplicial fan}

\subsection{Polytopes and fans}

We briefly recall basic definitions and properties of polyhedral fans and polytopes, and refer to~\cite{Ziegler-polytopes} for a classical textbook on this topic.

A hyperplane~$H \subset \R^d$ is a \defn{supporting hyperplane} of a set~$X \subset \R^d$ if~$H \cap X \ne \varnothing$ and~$X$ is contained in one of the two closed half-spaces of~$\R^d$ defined by~$H$.

We denote by~$\R_{\ge0}\rays \eqdef \set{\sum_{\ray \in \rays} \lambda_{\ray} \, \ray}{\lambda_{\ray} \in \R_{\ge0}}$ the \defn{positive span} of a set~$\rays$ of vectors of~$\R^d$.
A \defn{polyhedral cone} is a subset of~$\R^d$ defined equivalently as the positive span of finitely many vectors or as the intersection of finitely many closed linear halfspaces.
The \defn{faces} of a cone~$C$ are the intersections of~$C$ with the supporting hyperplanes of~$C$.
The $1$-dimensional (resp.~codimension~$1$) faces of~$C$ are called~\defn{rays} (resp.~\defn{facets}) of~$C$.
A cone is \defn{simplicial} if it is generated by a set of independent vectors.

A \defn{polyhedral fan} is a collection~$\Fan$ of polyhedral cones such that
\begin{itemize}
\item if~$C \in \Fan$ and~$F$ is a face of~$C$, then~$F \in \Fan$,
\item the intersection of any two cones of~$\Fan$ is a face of both.
\end{itemize}
A fan is \defn{simplicial} if all its cones are, \defn{complete} if the union of its cones covers the ambient space~$\R^d$, and \defn{essential} if it contains the cone~$\{\b{0}\}$.
% For two fans~$\Fan, \Fan[G]$ in~$\R^d$, we say that~$\Fan$ \defn{refines}~$\Fan[G]$ (and that~$\Fan[G]$ \defn{coarsens}~$\Fan$) if every cone of~$\Fan$ is contained in a cone of~$\Fan[G]$.

A \defn{polytope} is a subset~$P$ of~$\R^d$ defined equivalently as the convex hull of finitely many points or as a bounded intersection of finitely many closed affine halfspaces.
The \defn{dimension}~$\dim(P)$ is the dimension of the affine hull of~$P$.
The \defn{faces} of~$P$ are the intersections of~$P$ with its supporting hyperplanes.
The dimension~$0$ (resp.~dimension~$1$, resp.~codimension~$1$) faces are called \defn{vertices} (resp.~\defn{edges}, resp.~\defn{facets}) of~$P$.
A polytope is \defn{simple} if its supporting hyperplanes are in general position, meaning that each vertex is incident to $\dim(P)$ facets (or equivalently to $\dim(P)$ edges).

The (outer) \defn{normal cone} of a face~$F$ of~$P$ is the cone generated by the outer normal vectors of the facets of~$P$ containing~$F$.
In other words, it is the cone of vectors~$\b{c}$ such that the linear form~${\b{x} \mapsto \dotprod{\b{c}}{\b{x}}}$ on~$P$ is maximized by all points of the face~$F$.
The (outer) \defn{normal fan} of~$P$ is the collection of the (outer) normal cones of all its faces.
We say that a complete polyhedral fan~$\Fan$ in~$\R^d$ is \defn{polytopal} when it is the normal fan of a polytope~$P$ of~$\R^d$, and that~$P$ is a \defn{polytopal realization} of~$\Fan$.

\subsection{Type cone}

Fix an essential complete simplicial fan~$\Fan$ in~$\R^n$. Let~$\b{G}$ the $N \times n$-matrix whose rows are the rays of~$\Fan$ and let~$\b{K}$ be a $(N-n) \times N$-matrix that spans the left kernel of~$\b{G}$ (\ie $\b{K}\b{G} = 0$). For any height vector~$\b{h} \in \R^N$, we define the polytope
\[
P_\b{h} \eqdef \set{\b{x} \in \R^n}{\b{G}\b{x} \le \b{h}}.
\]
We say that $h$ is \defn{$\Fan$-admissible} if $P_\b{h}$ is a polytopal realization of~$\Fan$.
The following classical statement characterizes the $\Fan$-admissible height vectors.
It is a reformulation of regularity of triangulations of vector configurations, introduced in the theory of secondary polytopes~\cite{GelfandKapranovZelevinsky}, see also~\cite{DeLoeraRambauSantos}.
We present here a convenient formulation from~\cite[Lem.~2.1]{ChapotonFominZelevinsky}.

\begin{proposition}
\label{prop:characterizationPolytopalFan}
Then the following are equivalent for any height vector~$h \in \R^N$:
\begin{enumerate}
\item The fan~$\Fan$ is the normal fan of the polytope~$P_\b{h} \eqdef \set{\b{x} \in \R^n}{\b{G}\b{x} \le \b{h}}$.
\item For any two adjacent cones~$\R \rays$ and~$\R \rays'$ of~$\Fan$ with~$\rays \ssm \{\ray\} = \rays' \ssm \{\ray'\}$, we have
\[
\alpha \, h_{\ray} + \alpha' \, h_{\ray'} + \sum_{\ray[s] \in \rays \cap \rays'} \beta_{\ray[s]} \, h_{\ray[s]} > 0,
\]
where
\[
\alpha \, \ray + \alpha' \, \ray' + \sum_{\ray[s] \in \rays \cap \rays'} \beta_{\ray[s]} \, \ray[s] = 0
\]
is the unique (up to rescaling) linear dependence with~$\alpha, \alpha' > 0$ between the rays of~$\rays \cup \rays'$.
\end{enumerate}
\end{proposition}

\begin{notation}
For any two adjacent cones~$\R\rays$ and~$\R\rays'$ of~$\Fan$ with~$\rays \ssm \{\ray\} = \rays' \ssm \{\ray'\}$, we denote by~$\coefficient[{\ray[s]}][\rays][\rays']$ the coefficient of~$\ray[s]$ in the unique linear dependence between the rays of~$\rays \cup \rays'$, \ie such that
\[
\sum_{\ray[s] \in \rays \cup \rays'} \coefficient[{\ray[s]}][\rays][\rays'] \, \ray[s] = 0.
\]
These coefficients are \apriori{} defined up to rescaling, but we additionally fix the rescaling so that~${\coefficient[\ray][\rays][\rays'] + \coefficient[\ray'][\rays][\rays'] = 1}$.
\end{notation}

In this paper, we are interested in the set of all possible realizations of~$\Fan$. This was studied by P.~McMullen in~\cite{McMullen-typeCone}.

\begin{definition}
The \defn{type cone} of~$\Fan$ is the cone~$\typeCone(\Fan)$ of all $\Fan$-admissible height vectors~$\b{h}$:
\begin{align*}
\typeCone(\Fan) & \eqdef \set{\b{h} \in \R^N}{\Fan \text{ is the normal fan of } P_\b{h}} \\
& = \Bigset{\b{h} \in \R^N}{\sum_{\ray[s] \in \rays \cup \rays'} \coefficient[{\ray[s]}][\rays][\rays'] \, \b{h}_{\ray[s]} > 0 \text{ for any adjacent cones~$\R\rays, \R\rays'$ of~$\Fan$}}.
\end{align*}
\end{definition}

Note that the type cone is an open cone and contains a linearity subspace of dimension~$n$ (it is invariant by translation in~$\b{G} \R^n$). We could thus intersect~$\typeCone(\Fan)$ by the kernel of~$\b{G}$, or consider the projection~$\b{K}\typeCone(\Fan)$.

\begin{definition}
An \defn{extremal adjacent pair} of~$\Fan$ is a pair of adjacent cones~$\{\R\rays, \R\rays'\}$ of~$\Fan$ such that the corresponding inequality $\sum_{\ray[s] \in \rays \cup \rays'} \coefficient[{\ray[s]}][\rays][\rays'] \, \b{h}_{\ray[s]} > 0$ in the definition of the type cone~$\typeCone(\Fan)$ actually defines a facet of~$\typeCone(\Fan)$.
\end{definition}

In other words, extremal adjacent pairs define the extremal rays of the polar of the type cone~$\typeCone(\Fan)$.
Understanding the extremal adjacent pairs of~$\Fan$ enables to describe its type cone~$\typeCone(\Fan)$ and thus all its polytopal realizations.

\subsection{Coherent fans}

Two rays~$\ray$ and~$\ray'$ of~$\Fan$ are called \defn{compatible} if there are both contained in a cone of~$\Fan$ and \defn{exchangeable} if there are two adjacent cones~$\R\rays$ and~$\R\rays'$ of~$\Fan$ with ${\rays \ssm \{\ray\} = \rays' \ssm \{\ray'\}}$.
If~$\rays(\Fan)$ denotes the set of rays of~$\Fan$, a \defn{compatibility degree} for~$\Fan$ is a function~$\compatibilityDegree{-}{-}: \rays(\Fan) \times \rays(\Fan) \to \R$ such that
\begin{itemize}
\item $\compatibilityDegree{\ray}{\ray'} = -1$ if~$\ray = \ray'$ and is non-negative otherwise,
\item $\compatibilityDegree{\ray}{\ray'} = 0$ if ~$\ray$ and~$\ray'$ are compatible,
\item $\compatibilityDegree{\ray}{\ray'} > 0$ if~$\ray \ne \ray'$ are incompatible,
\item $\compatibilityDegree{\ray}{\ray'} = 1 = \compatibilityDegree{\ray'}{\ray}$ if and only if~$\ray$ and~$\ray'$ are exchangeable.
\end{itemize}

We say that the fan~$\Fan$ is \defn{coherent} if for any two exchangeable rays~$\ray, \ray'$ of~$\Fan$, the linear dependence
\vincent{Not sure of the choice of the name.}
\[
\sum_{\ray[s] \in \rays \cup \rays'} \coefficient[{\ray[s]}][\rays][\rays'] \, \ray[s] = 0.
\]
does not depend on the pair~$\{\rays, \rays'\}$ of adjacent cones of~$\Fan$ with $\rays \ssm \{\ray\} = \rays' \ssm \{\ray'\}$, but only on the pair of rays~$\ray, \ray'$.
This implies in particular that the rays~$\ray[s]$ for which~$\coefficient[{\ray[s]}][\rays][\rays'] \ne 0$ belong to~$\rays \cup \rays'$ for any pair~$\{\rays, \rays'\}$ of adjacent cones of~$\Fan$ with $\rays \ssm \{\ray\} = \rays' \ssm \{\ray'\}$. These rays are thus called \defn{forced rays} for the exchangeable pair~$\{\ray, \ray'\}$.

When~$\Fan$ is coherent, we change the notation~$\coefficient[{\ray[s]}][\rays][\rays']$ to~$\coefficient$ and we obtain that the type cone of~$\Fan$ is expressed as
\[
\typeCone(\Fan) = \Bigset{\b{h} \in \R^N}{\sum_{\ray[s] \in \rays \cup \rays'} \coefficient \, \b{h}_{\ray[s]} > 0 \text{ for any exchangeable rays~$\ray, \ray'$ of~$\Fan$}}.
\]

\begin{definition}
In a coherent fan~$\Fan$, an \defn{extremal exchangeable pair} is a pair of exchangeable rays~$\{\ray,\ray'\}$ such that the corresponding inequality~${\sum_{\ray[s] \in \rays \cup \rays'} \coefficient \, \b{h}_{\ray[s]} > 0}$ defines a facet of the type cone~$\typeCone(\Fan)$.
\end{definition}

In this paper, we will only consider coherent fans, and our objective will be to describe their extremal exchangeable pairs.

\subsection{Alternative polytopal realizations}

In this section, we provide alternative polytopal realizations of the fan~$\Fan$.
We also discuss the behavior of these realizations in the situation when the type cone~$\typeCone(\Fan)$ is simplicial.

We still consider an essential complete simplicial fan~$\Fan$ in~$\R^n$, the $N \times n$-matrix~$\b{G}$ whose rows are the rays of~$\Fan$, and the $(N-n) \times N$-matrix~$\b{K}$ which spans the left kernel of~$\b{G}$.

\begin{proposition}
\label{prop:alternativePolytopalRealization}
The affine map~$\Psi: \R^n \to \R^N$ defined by~$\Psi(\b{x}) = \b{h} - G\b{x}$ sends the polytope
\[
P_\b{h} \eqdef \set{\b{x} \in \R^n}{\b{G}\b{x} \le \b{h}}
\]
to the polytope
\[
Q_\b{h} \eqdef \set{\b{z} \in \R^N}{\b{K}\b{z} = \b{K}\b{h} \text{ and } \b{z} \ge 0}.
\]
\end{proposition}

\begin{proof}
For~$\b{x}$ in~$P_\b{h}$, we have $\Psi(\b{x}) \ge 0$ by definition and~$\b{K}\Psi(\b{x}) = \b{K}\b{h} - \b{K}\b{G}\b{x} = \b{K}\b{h}$ since~$\b{K}$ is the left kernel of~$\b{G}$. Therefore~$\Psi(\b{x}) \in Q_\b{h}$.
Moreover, the map~$\Psi : P_\b{h} \to Q_\b{h}$ is:
\begin{itemize}
\item injective: Indeed, $\Psi(\b{x}) = \Psi(\b{x}')$ implies~$\b{G}(\b{x} - \b{x}') = 0$ and~$\b{G}$ has full rank since~$\Fan$ is essential and complete.
\item surjective: Indeed, for~$\b{z} \in Q_\b{h}$, we have~$\b{K}(\b{h}-\b{z}) = 0$ so that~$\b{h}-\b{z}$ belongs to the right kernel of~$\b{K}$ which is the image of~$\b{G}$. Therefore, there exists~$\b{x} \in \R^n$ such that~$\b{h}-\b{z} = \b{G}\b{x}$. Therefore, $\b{z} = \Psi(\b{x}$ and~$\b{x} \in P_\b{h}$ since~$\b{h} - \b{G}\b{x} = \b{z} \ge 0$.\qedhere
\end{itemize}
\end{proof}

\begin{corollary}
Assume that the type cone~$\typeCone(\Fan)$ is simplicial and let~$\b{K}$ be the $(N-n) \times N$-matrix whose rows are the outer normal vectors of the facets of~$\typeCone(\Fan)$. Then the polytope
\[
R_\b{\ell} \eqdef \set{\b{z} \in \R^N}{K\b{z} = \b{\ell} \text{ and } \b{z} \ge 0}
\]
is a realization of the fan~$\Fan$ for any positive vector~$\b{\ell} \in \R_{>0}^{N-n}$.
Moreover, the polytopes~$R_\b{\ell}$ for~$\b{\ell} \in \R_{>0}^{N-n}$ describe all polytopal realizations of~$\Fan$.
\end{corollary}

\begin{proof}
Let~$\b{\ell} \in \R_{>0}^{N-n}$.
Since~$\b{K}$ has full rank there exists~$\b{h} \in \R^N$ such that~$\b{K}\b{h} = \b{\ell}$.
Since~$\b{K}\b{h} \ge 0$ and the rows of~$\b{K}$ are precisely all outer normal vectors of the facets of the type cone~$\typeCone(\Fan)$, we obtain that~$\b{h}$ belongs to~$\typeCone(\Fan)$.
Since $R_\b{\ell} = Q_\b{h} \sim P_\b{h}$ by \cref{prop:alternativePolytopalRealization}, we conclude that~$R_\b{\ell}$ is a polytopal realization of~$\Fan$.
Since~$\typeCone(\Fan)$ is simplicial, we have~$\b{K}\typeCone(\Fan) = \R_{>0}^{N-n}$, so that we obtain all polytopal realizations of~$\Fan$ this way.
\end{proof}


\section{Applications to different generalizations of the associahedron}

\subsection{Generalized associahedra}

The acyclic and simply-laced case was treated in~\cite{BazierMatteDouvilleMousavandThomasYildirim}.
Computer experiments indicate that the type cone is always simplicial for any seed (acyclic or not) in any finite type cluster algebra.
While the case of acyclic seeds can be handled by representation theory~\cite{BazierMatteDouvilleMousavandThomasYildirim}, we have no proof at the moment for cyclic seeds.

One important observation is that it seems there is one extremal exchangeable pair for each positive $c$-vector, meaning that for each positive $c$-vector~$\beta$, there is precisely one extremal exchangeable pair~$\{x,x'\}$ of cluster variables for which the flip of~$x$ to~$x'$ (for any pair of clusters~$\{X,X'\}$ with~$X \ssm \{x\} = X' \ssm \{x'\}$) is in the direction of~$\beta$.
The goal is thus to determine for each positive root which exchangeable pair is extremal.
This should be done using the Auslander-Reiten quiver to construct two cluster variables from a $c$-vector (see the next paragraph for the idea).

\subsection{Gentle associahedra}

We are already missing a criterion for exchangeable pairs of walks. See \cite[Sect.~9]{BrustleDouvilleMousavandThomasYildirim}.
It seems that two exchangeable pairs are always kissing along a single distinguishable string.
We need to prove that to see that the non-kissing fan is coherent (because for a pair of exchangeable walks, I can completely reconstruct the $\b{g}$-vector dependence of the flip if I know there distinguished substring).
Note however that
\begin{enumerate}
\item exchangeable pairs might kiss along additional non-distinguishable strings (example: just a loop),
\item non-exchangeable walks might kiss along a single distinguishable string (example: see the cyclic triangle in~\cite{PaluPilaudPlamondon}).
\end{enumerate}

We have some conjectures on what the extremal exchangeable pairs should be.
One clear (but a bit empty) result is that extremal exchangeable pairs correspond in the Auslander-Reiten quiver to rectangles whose four vertices are non-self-kissing and that cannot be tiled with smaller such rectangles.

We checked on some (but probably not enough) small gentle quivers the following properties:
\begin{enumerate}
\item Any $c$-vector (\ie distinguishable string) is the direction of at least one extremal exchangeable pair.
\item Consider a distinguishable string~$\sigma$. Let~$\omega$ (resp.~$\omega'$) be the walk obtained from~$\sigma$ by adding two hooks (resp.~two cohooks) at the endpoints of~$\sigma$. If the walks~$\omega$ and~$\omega'$ are non-self-kissing and exchangeable, then they form the unique extremal exchangeable pair directed by~$\sigma$. These extremal exchangeable pairs correspond to meshes of the Auslander-Reiten quiver.
\item Otherwise, $\sigma$ is the direction to one or more extremal exchangeable pairs obtained by moving further in the Auslander-Reiten quiver (this is really unclear at the moment).
\end{enumerate}


\subsection{Graph associahedra}

Let~$\graphG$ be a graph with vertex set~$\ground$.
Let~$\connectedComponents(\graphG)$ denote the set of connected components of~$\graphG$ and define~$n \eqdef |\ground|-|\connectedComponents(\graphG)|$.
A \defn{tube} of~$\graphG$ is a non-empty connected induced subgraph of~$\graphG$.
The set of tubes of~$\graphG$ is denoted by~$\tubes(\graphG)$.
The inclusion maximal tubes of~$\graphG$ are its connected components~$\connectedComponents(\graphG)$, and all other tubes are called \defn{proper}.
Two tubes~$\tube, \tube'$ of~$\graphG$ are \defn{compatible} if they are either nested (\ie $\tube \subseteq \tube'$ or~$\tube' \subseteq \tube$), or disjoint and non-adjacent (\ie $\tube \cup \tube'$ is not a tube of~$\graphG$).
A \defn{tubing} on~$\graphG$ is a set~$\tubing$ of pairwise compatible proper tubes of~$\graphG$.
The collection of all tubings on~$\graphG$ is a simplicial complex, called \defn{nested complex} of~$\graphG$ and denoted by~$\nestedComplex(\graphG)$.

Let~$(\b{e}_v)_{v \in \ground}$ be the canonical basis of~$\R^\ground$, let~${\HH \eqdef \bigset{\b{x} \in \R^\ground}{\sum_{w \in W} x_w = 0 \text{ for all } W \in \connectedComponents(\graphG)}}$ and~$\pi : \R^\ground \to \HH$ denote the orthogonal projection on~$\HH$.
Let~$\gvector{\tube} \eqdef \pi \big( \sum_{v \in \tube} \b{e}_v \big)$ denote the projection of the characteristic vector of a tube~$\tube$ of~$\graphG$, and define~$\gvectors{\tubing} \eqdef \set{\gvector{\tube}}{\tube \in \tubing}$ for a tubing~$\tubing$ on~$\graphG$.
Note that~$\gvector{\graphG} = 0$ by definition.
These vectors support a complete simplicial fan realization of the nested complex:

\begin{theorem}[\cite{CarrDevadoss, Postnikov, FeichtnerSturmfels, Zelevinsky}]
\label{theo:gFan}
For any graph~$\graphG$, the collection of cones
\[
\nestedFan(\graphG) \eqdef \set{\R_{\ge 0} \, \gvectors{\tubing}}{\tubing \text{ tubing on } \graphG}
\]
is a complete simplicial fan of~$\HH$, called \defn{nested fan} of~$\graphG$, which realizes the nested complex~$\nestedComplex(\graphG)$.
\end{theorem}

It is proved in~\cite{CarrDevadoss, Devadoss, Postnikov, FeichtnerSturmfels, Zelevinsky} that the nested fan comes from a polytope. 

\begin{theorem}[\cite{CarrDevadoss, Devadoss, Postnikov, FeichtnerSturmfels, Zelevinsky}]
\label{theo:graphAssociahedron}
For any graph~$\graphG$, the nested fan~$\nestedFan(\graphG)$ is the normal fan of the graph associahedron~$\Asso[\graphG]$.
\end{theorem}

The following statement follows from~\cite{MannevillePilaud-compatibilityFans, Zelevinsky}.

\begin{proposition}
\label{prop:exchangeablePairsGA}
Let~$\tube, \tube'$ be two tubes of~$\graphG$. Then
\begin{enumerate}[(i)]
\item The tubes~$\tube$ and~$\tube'$ are exchangeable if and only if $\tube'$ has a unique neighbor~$v$ in~$\tube \ssm \tube'$ and $\tube$ has a unique neighbor~$v'$ in~$\tube' \ssm \tube$.
\item For any maximal tubings~$\tubing, \tubing$ on~$\graphG$ with $\tubing \ssm \{\tube\} = \tubing' \ssm \{\tube'\}$, both~$\tubing \cup \connectedComponents(\graphG)$ and~$\tubing' \cup \connectedComponents(\graphG)$ contain the tube~$\tube \cup \tube'$ and the connected components of~$\tube \cap \tube'$.
\item For any maximal tubings~$\tubing, \tubing$ on~$\graphG$ with $\tubing \ssm \{\tube\} = \tubing' \ssm \{\tube'\}$, the linear dependence between the $\b{g}$-vectors of~$\tubing \cup \tubing'$ is given by
\[
\gvector{\tube} + \gvector{\tube'} = \gvector{\tube \cup \tube'} + \sum_{\tube[s] \in \connectedComponents(\tube \cap \tube')} \gvector{\tube[s]}.
\]
In other words, the nested fan of~$\graphG$ is coherent.
\item For any maximal tubings~$\tubing, \tubing$ on~$\graphG$ with $\tubing \ssm \{\tube\} = \tubing' \ssm \{\tube'\}$, the $\b{c}$-vector orthogonal to all $\b{g}$-vectors~$\gvector{\tube[s]}$ for~$\tube[s] \in \tubing \cap \tubing'$ is~$\b{e}_v - \b{e}_{v'}$.
\end{enumerate}
\end{proposition}

\begin{proof}
Points~(i) and~(ii) was proved in~\cite{MannevillePilaud-compatibilityFans}. Point~(iii) follows from the fact that
\[
\sum_{v \in \tube} \b{e}_v + \sum_{v \in \tube'} \b{e}_v = \sum_{v \in \tube \cup \tube'} \b{e}_v + \sum_{v \in \tube \cap \tube'} \b{e}_v = \sum_{v \in \tube \cup \tube'} \b{e}_v + \sum_{\substack{\tube[s] \in \connectedComponents(\tube \cap \tube') \\ v \in \tube[s]}} \b{e}_v.
\]
Finally, any tube~$\tube[s] \in \tubing \cap \tubing'$ that contains~$v$ or~$v'$ actually contains both (to be compatible with~$\tube$ and~$\tube'$). Therefore, $\gvector{\tube[s]}$ is orthogonal to~$\b{e}_v - \b{e}_{v'}$ for any tube~$\tube[s] \in \tubing \cap \tubing'$.
\end{proof}

We are now ready to describe the type cone of the nested fan~$\nestedFan(\graphG)$.
We start with the case of the path.
Recall that the tubes of the path are the intervals of~$[n]$ and that two intervals~$[i,j]$ and~$[i',j']$ of~$[n]$ are exchangeable if either~${i < i' \le j+1 < j'+1}$ or~${i' < i \le j'+1 < j+1}$.

\begin{proposition}
\label{prop:extremalExchangeablePairsA}
Two intervals~$[i,j]$ and~$[i',j']$ of~$[n]$ form an extremal exchangeable pair for the nested fan the path if and only if~$i = i'+1$ and~$j = j'+1$, or the opposite.
\end{proposition}

\begin{proof}
Let~$(\b{f}_{[i,j]})_{1 \le i \le j \le n}$ denote the canonical basis of~$\R^N$.
Consider two exchangeable tubes~$[i,j]$ and~$[i',j']$ with~$i < i' \le j+1 < j'+1$.
By \cref{prop:exchangeablePairsGA}\,(iii), the linear dependence between the corresponding $\b{g}$-vectors is given by
\[
\gvector{[i,j]} + \gvector{[i',j']} = \gvector{[i,j']} + \gvector{[i',j]}
\]
Therefore, the outer normal vector of the corresponding inequality of the type cone~$\typeCone(\nestedFan(\graphG))$ is given by
\[
\b{n}(i,j,i',j') \eqdef \b{f}_{[i,j]} + \b{f}_{[i',j']} - \b{f}_{[i,j']} - \b{f}_{[i',j]}
\]
Denoting
\[
\b{m}(k,\ell) \eqdef \b{n}(k,\ell-1,k+1,\ell) = \b{f}_{[k, \ell-1]} + \b{f}_{[k+1, \ell]} - \b{f}_{[k, \ell]} - \b{f}_{[k+1, \ell-1]},
\]
we obtain that
\[
\b{n}(i,j,i',j') = \sum_{\substack{k \in {[i,i'[} \\ \ell \in {]j,j']}}} \b{m}(k,\ell).
\]
Indeed, on the right hand side, the basis vector~$\b{f}_{[k,\ell]}$ appears with a positive sign in~$\b{m}(k,{\ell+1})$ for~$(k,\ell) \in {[i,i'[} \times {[j,j'[}$ and in~$\b{m}(k-1, \ell)$ for~$(k,\ell) \in {]i,i']} \times {]j,j']}$, and with a negative sign in~$\b{m}(k,\ell)$ for~$(k,\ell) \in {[i,i'[} \times {]j,j']}$ and in~$\b{m}(k-1,\ell+1)$ for~$(k,\ell) \in {]i,i']} \times {[j,j'[}$.
Therefore, these contributions all vanish except when~$[k, \ell]$ is one of the tubes~$[i,j]$, $[i',j']$, $[i,j']$ or~$[i',j]$.
%For any~$i \le k < i'$ and~$j < \ell \le j'$, the exchange relation between the tubes~$[k, \ell-1]$ and~$[k+1, \ell]$ is given by
%\begin{equation}
%\label{eq:exch2}
%\tag{$\mathrm{ER}_k^\ell$}
%\gvector{[k, \ell-1]} + \gvector{[k+1, \ell]} = \gvector{[k, \ell]} + \gvector{[k+1, \ell-1]}.
%\end{equation}
%Summing the exchange relations~\eqref{eq:exch2} over all pairs~$(k,\ell) \in {[i,i'[} \times {]j,j']}$, we obtain after simplifications the exchange relation~\eqref{eq:exch1}.
%Indeed, the $\b{g}$-vector of each tube~$[k, \ell]$ appears on the left hand sides of the exchange relations~($\mathrm{ER}_k^{\ell+1}$) for~$(k,\ell) \in {[i,i'[} \times {[j,j'[}$ and ($\mathrm{ER}_{k-1}^\ell$) for~$(k,\ell) \in {]i,i']} \times {]j,j']}$ and on the right hand sides of the exchange relations~($\mathrm{ER}_k^\ell$) for~$(k,\ell) \in {[i,i'[} \times {]j,j']}$ and~($\mathrm{ER}_{k-1}^{\ell+1}$) for~$(k,\ell) \in {]i,i']} \times {[j,j'[}$.
%Therefore, these contributions all vanish except when~$[k, \ell]$ is one of the tubes~$[i,j]$, $[i',j']$, $[i,j']$ or~$[i',j]$.
This shows that any exchange relation is a positive linear combination of the exchange relations corresponding to all pairs of tubes~$[i,j]$ and~$[i',j']$ of~$[n]$ such that~$i = i'+1$ and~$j = j'+1$.

We now need to show that all these exchangeable pairs are extremal.
Assume that~$\b{m}(k,\ell)$ can be written as the linear combination
\begin{equation}
\label{eq:linComb}
\b{m}(k,\ell) = \sum \lambda(i,j,i',j') \, \b{n}(i,j,i',j'),
\end{equation}
where~$\lambda(i,j,i',j') \ge 0$ for all exchangeable pairs~$\{[i,j], [i',j']\}$.
Note that~$\sum \lambda(i,j,i',j') > 1$ since the coefficient of~$\b{f}_{[k,\ell]}$ in~$\b{m}(k,\ell)$ is~$-1$.
Consider the linear form~$\Phi: \R^N \to \R$ defined by~$\Phi(\b{f}_{[i,j]}) = -(j-i)^2$.
A quick computation shows that
\[
\Phi(\b{n}(i,j,i',j')) = 2(i'-i)(j'-j).
\]
Applying~$\Phi$ to \cref{eq:linComb}, we obtain
\[
1 = \sum \lambda(i,j,i',j')(i'-i)(j'-j).
\]
Since~$\sum \lambda(i,j,i',j') > 1$ we obtain that~$\lambda(k,\ell-1,k+1,\ell-1) = 1$ and~$\lambda(i,j,i',j') = 0$ for all other exchangeable pairs~$\{[i,j], [i',j']\}$.
\end{proof}

\cref{prop:extremalExchangeablePairsA} extends to arbitrary graphs as follows.
% We denote by~$X \symdif Y$ the symmetric difference of two sets~$X$ and~$Y$.

\begin{proposition}
\label{prop:extremalExchangeablePairsGA}
%A pair~$\{\tube, \tube'\}$ of exchangeable tubes of~$\graphG$ is extremal if and only if~$|\tube \symdif \tube'| = 2$.
Two tubes~$\tube$ and~$\tube'$ of~$\graphG$ form an extremal exchangeable pair for the nested fan of~$\graphG$ if and only if~${\tube \ssm \{v\} = \tube' \ssm \{v'\}}$ for some neighbor~$v$ of~$\tube'$ and some neighbor~$v'$ of~$\tube$.
\end{proposition}

\begin{proof}
Consider an exchangeable pair~$\{\tube, \tube'\}$ of tubes of~$\graphG$ and let~$p = |\tube \cup \tube'|$.
By \cref{prop:exchangeablePairsGA}, $\tube'$ has a unique neighbor~$v$ in~$\tube \ssm \tube'$ and $\tube$ has a unique neighbor~$v'$ in~$\tube' \ssm \tube$.
Therefore, $\tube \ssm \tube'$ and~$\tube' \ssm \tube$ are both connected.
We can thus label the elements of~$\tube \cup \tube'$ by~$\{v_1, \dots, v_p\}$ such that~$\{v_i, \dots, v_j\}$ induces a tube~$\tube_{k,\ell}$ of~$\graphG$ for any~$1 \le k \le \ell \le p$. 
The map~$[k,\ell] \mapsto \tube_{k,\ell}$ is thus an injection from the tubes of the path~$\graphG[P]_p$ to that of~$\graphG$ that fulfills
\begin{itemize}
\item $\tube_{k,\ell-1} \ssm \{v_k\} = \tube_{k+1,\ell} \ssm \{v_\ell\}$ for any~$1 \le k \le \ell \le p$, and
\item $\gvector{\tube_{k,\ell}} = M\gvector{[k,\ell]}$ for any~$1 \le k \le \ell \le p$, where~$M$ is the matrix sending~$\b{e}_m$ to~$\b{e}_{v_m}$.
\end{itemize}
Therefore, this map transports the linear combinations of \cref{prop:extremalExchangeablePairsA}.
We conclude that the exchange relation corresponding to the exchangeable pair~$\{\tube, \tube'\}$ is a positive linear combination of the exchange relations corresponding to the pairs~$\{\tube_{k,\ell-1}, \tube_{k+1,\ell}\}$ for all~${(k,\ell) \in {[1, p-|\tube'|+1[} \times {]|\tube|, p]}}$.

We now need to show that all these pairs are extremal.
\vincent{TODO}
\end{proof}

\begin{corollary}
For any $\b{c}$-vector supports at least one extremal exchangeable pair.
\end{corollary}

\begin{proof}
Consider a $\b{c}$-vector~$\b{e}_v - \b{e}_{v'}$ for two distinct vertices~$v, v'$ in a common connected component of~$\graphG$. Let~$\tube[s]$ be a path from~$v$ to~$v'$ in~$\graphG$ and let~$\tube \eqdef \tube[s] \cup \{v\}$ and~$\tube' \eqdef \tube[s] \cup \{v'\}$. Then $\{\tube, \tube'\}$ is an extremal exchangeable pair with $\b{c}$-vector~$\b{e}_v - \b{e}_{v'}$.
\end{proof}

\begin{corollary}
Consider a graph~$\graphG$ on~$\ground$ with tubes~$\tubes(\graphG)$ and a height vector~$\b{h} \in \R^{\tubes(\graphG)}$ such that~$\b{h}_{\graphG} = 0$ and
\[
\b{h}_{\tube} + \b{h}_{\tube'} = \b{h}_{\tube \cup \tube'} + \sum_{\tube[s] \in \connectedComponents(\tube \cap \tube')} \b{h}_{\tube[s]}
\]
for any pair of tubes~$\{\tube, \tube'\}$ of~$\graphG$ with~$\tube \ssm \{v\} = \tube' \ssm \{v'\}$ where~$v$ is a neighbor of~$\tube'$ and~$v'$ is a neighbor of~$\tube$.
Then the nested fan~$\nestedFan(\graphG)$ is the normal fan of the graph associahedron
\[
\set{\b{x} \in \R^\ground}{\dotprod{\gvector{\tube}}{\b{x}} \le \b{h}_{\tube} \text{ for any tube } \tube \in \tubes(\graphG)}.
\]
\end{corollary}

We now compute the number of extremal exchangeable pairs of the nested fan.
For a tube~$\tube$ of~$\graphG$, we denote by~$\nonDisconnecting(\tube)$ the number of non-disconnecting vertices of~$\tube$.
In other words, it is the number of tubes covered by~$\tube$ in the inclusion poset of all tubes of~$\graphG$.

\begin{proposition}
\label{prop:numberExtremalExchangeablePairsGA}
The nested fan~$\nestedFan(\graphG)$ has~$\sum_{\tube \in \tubes(\graphG)} \binom{\nonDisconnecting(\tube)}{2}$ extremal exchangeable pairs.
\end{proposition}

\begin{proof}
By \cref{prop:extremalExchangeablePairsGA}, extremal exchangeable pairs are in bijection with triples~$(\tube, \tube[s], \tube[s'])$ of tubes of~$\graphG$ where~$\tube[s]$ and~$\tube[s']$ are distinct tubes covered by~$\tube$ in the inclusion poset of all tubes of~$\graphG$.
The result immediately follows.
\end{proof}

The formula of \cref{prop:numberExtremalExchangeablePairsGA} can be made more explicit for specific families of graph associahedra

\begin{proposition}
The number of extreme exchangeable pairs of the nested fan~$\nestedFan(\graphG)$ is:
\begin{itemize}
\item $2^{n-2}\binom{n}{2}$ for the permutahedron (complete graph associahedron),
\item $\binom{n}{2}$ for the associahedron (path associahedron),
\item $3\binom{n}{2} - n$ for the cyclohedron (cycle associahedron),
\item $n-1+2^{n-3}\binom{n-1}{2}$ for the stellohedron (star associahedron).
\end{itemize}
\end{proposition}

\begin{proof}
For the permutahedron, choose any two vertices~$v,v'$, and complete them into a tube by selecting any subset of the~$n-2$ remaining vertices.
For the associahedron, choose any two vertices~$v,v'$, and complete them into a tube by taking the path between them.
For the cyclohedron, choose the two vertices~$v,v'$, and complete them into a tube by taking either all the cycle, or one of the two paths between~$v$ and~$v'$ (this gives three options in general, but only two when~$v,v'$ are neighbors).
For the stellohedron, choose either~$v$ as the center of the star and~$v'$ as one of the $n-1$ leaves, or $v$~and~$v'$ as leaves of the star and complete them into a tube by taking the center and any subset of the $n-3$ remaining leaves.
\end{proof}

%\begin{example}
%The formula of \cref{prop:numberExtremalExchangeablePairsGA} can be made more explicit for specific families of graph associahedra:
%\begin{itemize}
%\item the permutahedron (complete graph associahedron) has~$2^{n-2}\binom{n}{2}$ extreme exchangeable pairs (choose the two vertices~$v,v'$, and complete them into a tube by selecting any subset of the~$n-2$ remaining vertices),
%\item the associahedron (path associahedron) has~$\binom{n}{2}$ extreme exchangeable pairs (choose the two vertices~$v,v'$, and complete them into a tube by taking the path between them),
%\item the cyclohedron (cycle associahedron) has~$3\binom{n}{2} - n$ extreme exchangeable pairs (choose the two vertices~$v,v'$, and complete them into a tube by taking either all the cycle, or one of the two paths between~$v$ and~$v'$; this gives three options in general, but only two when~$v,v'$ are neighbors),
%\item the stellohedron (star associahedron) has $n-1+2^{n-3}\binom{n-1}{2}$ extreme exchangeable pairs (choose either~$v$ as the center of the star and~$v'$ as one of the $n-1$ leaves, or $v$~and~$v'$ as leaves of the star and complete them into a tube by taking the center and any subset of the $n-3$ remaining leaves).
%\end{itemize}
%\end{example}

To conclude on graph associahedra, we characterize the graphs~$\graphG$ whose nested fan has a simplicial type cone.

\begin{proposition}
The type cone~$\typeCone(\nestedFan(\graphG))$ is simplicial if and only if~$\graphG$ is a path.
\end{proposition}

\begin{proof}
Note that any tube~$\tube$ with~$|\tube| \ge 2$ has two non-disconnecting vertices when it is a path, and at least three non-disconnecting vertices otherwise (the leaves of an arbitrary spanning tree of~$\tube$).
Therefore, each tube of~$\tubes(\graphG)$ which is not a singleton contributes to at least one extremal exchangeable pairs.
We conclude that the number of extremal exchangeable pairs is at least:
\[
|\tubes(\graphG)| - |V| = |\tubes(\graphG) \ssm \{\graphG\}| - (|V|-1) = N - n,
\]
with equality if and only if all tubes of~$\graphG$ are paths, \ie if and only if~$\graphG$ is a path.
\end{proof}

\subsection{Quotientopes}

Be careful that the quotient fans are not simplicial.

\subsection{Brick polytopes}

Be careful that the brick fans are not simplicial.

%%%%%%%%%%%%%%%%%%%%%%%%%%%%%%%%%%%%%%%

\bibliographystyle{alpha}
\bibliography{typeConeAssociahedra}
\label{sec:biblio}

\end{document}
